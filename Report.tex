\documentclass{article}
\usepackage[utf8]{inputenc}
%\usepackage[titletoc]{appendix}
\usepackage[english]{babel}
\usepackage{graphicx}
\usepackage{amsmath}
\usepackage{amsthm}
\usepackage{amssymb}
\usepackage{mathtools}         
\DeclarePairedDelimiter\Floor\lfloor\rfloor

\newtheorem{theorem}{Theorem}[section]
\newtheorem{corollary}{Corollary}[theorem]
\newtheorem{lemma}[theorem]{Lemma}
\newtheorem*{remark}{Remark}

\begin{document}

\title{Properties of the Temperature Distribution Curves of a Metal Bar}
\author{Anand Balivada \\ Aryan Mishra}
\maketitle

\begin{abstract}
 We investigate the temperature distribution curves of a metal bar, heated at the centre for a specified time interval, and experimentally establish the conservation of the area under the graphs of these curves. We proceed to compare these to numerical simulations for a one dimensional bar under similar conditions, but with an imposed Neumann Boundary condition to represent insulation (ensuring conservation of area under the graphs). 
\end{abstract}

\tableofcontents

\section{Introduction}


\section{Theory and Problem Statement}
In a metal bar of length $l$ treated as a one dimensional, closed subset $[0,l]$ of $\mathbb{R}$, the distribution of temperature $T(x, t), x \in [0,l], t > 0$, obeys the equation: $$\frac{\partial T}{\partial t} = \alpha\frac{\partial^{2}T}{\partial x^{2}} \dots (1)$$
where $\alpha$ is the thermal diffusivity of the material of the bar given by $\frac{k}{c\rho}$, where $k,c,\rho$ are the thermal conductivity, specific heat capacity and mass density of the metal bar respectively (Chapter 1.6 of [1]). \\ 
We enforce the Neumann Boundary Condition $\frac{\partial T}{\partial x}(0,t) = \frac{\partial T}{\partial x}(l,t)  = 0$, whose physical significance is the insulation of the sides of the bar. 
\begin{remark}
The heat equation neglects the loss of heat through convection or radiation, and thus the heat equation where the sides of the bar are insulated guarantees no loss of heat.
\end{remark}
\begin{theorem}
$\int_{0}^{l} T(x,t) dx$ is constant for all times $t > 0$. 
\end{theorem}

$\textbf{Problem Statement}$: To experimentally verify $Theorem \text{ }2.1$ by measuring the measuring the temperature distribution of a metal bar. 

\section{Design of the Experiment}



\section{Results}


\section{Conclusions}

\section{References}
[1] Churchill, Ruel.V., Ward, James B., Fourier Series and Boundary Value Problems.


%\begin{appendices}
% \section{Simulation Code in GNU Octave}
% \section{Code for Arduino Genuino with Thermocouple}
%\end{appendices}

\end{document}